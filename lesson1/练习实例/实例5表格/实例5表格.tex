\documentclass{article}
\usepackage[UTF8]{ctex} % 支持中文

\title{LaTeX表格练习}
\author{刘浩洋 24040021022}
\date{2025年8月31日}

\begin{document}

\maketitle

% --- 基本表格 ---
\section{基本表格结构}
最简单的表格由 \texttt{tabular} 环境构成。列格式在花括号内定义:
\begin{itemize}
    \item \texttt{l}: 左对齐 (Left)
    \item \texttt{c}: 居中对齐 (Center)
    \item \texttt{r}: 右对齐 (Right)
    \item \texttt{|}: 竖线分隔符 (Vertical line)
\end{itemize}

% 示例1:无边框表格
\begin{tabular}{lcr}
姓名 & 年龄 & 城市 \\
张三 & 20 & 北京 \\
李四 & 22 & 上海 \\
王五 & 19 & 广州 \\
\end{tabular}

% --- 带边框的表格 ---
\section{带边框的表格}
通过在列格式中添加 \texttt{|} 和使用 \texttt{\textbackslash hline} 命令,可以为表格添加边框。

% 示例2:带完整边框的表格
\begin{tabular}{|l|c|r|}
\hline
\textbf{姓名} & \textbf{年龄} & \textbf{城市} \\
\hline
张三 & 20 & 北京 \\
\hline
李四 & 22 & 上海 \\
\hline
王五 & 19 & 广州 \\
\hline
\end{tabular}

% --- 复杂表格 ---
\section{复杂表格示例}
可以使用 \texttt{\textbackslash cline} 创建部分横线,实现更复杂的表格结构。

% 示例3:部分横线的表格
\begin{tabular}{|c|c|c|c|}
\hline
\multicolumn{2}{|c|}{\textbf{分数组}} & \multicolumn{2}{c|}{\textbf{排名}} \\
\hline
数学 & 英语 & 数学 & 英语 \\
\hline
95 & 88 & 1 & 5 \\
\cline{1-2}
85 & 92 & 3 & 2 \\
\hline
\end{tabular}

% --- 说明 ---
% \multicolumn{2}{|c|}{...} 表示合并2列,居中对齐,有左右竖线。
% \cline{1-2} 表示在第1列到第2列之间画一条横线。

\end{document}