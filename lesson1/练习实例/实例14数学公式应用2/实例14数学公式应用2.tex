\documentclass{article}
\usepackage[UTF8]{ctex} % 支持中文
\usepackage{amsmath} % 数学公式核心
\usepackage{amsthm} % 定理环境
\usepackage{amssymb} % 额外数学符号

\title{LaTeX数学公式综合练习2}
\author{刘浩洋 24040021022}
\date{2025年9月1日}

% --- 定义定理环境 ---
\newtheorem{theorem}{定理}[section] % 按章节编号
\newtheorem{lemma}[theorem]{引理}   % 与定理共享编号
\newtheorem{corollary}[theorem]{推论}
\newtheorem{definition}{定义}[section]
\newtheorem{example}{例}[section]
\newtheorem{remark}{注}[section]

\begin{document}

\maketitle

% --- 集合论与逻辑 ---
\section{集合论与逻辑符号}
LaTeX提供了丰富的符号用于表达集合和逻辑关系。

\begin{itemize}
    \item \textbf{集合符号}:
    $A = \{1, 2, 3\}$, $x \in A$, $B \subset A$, $A \cup B$, $A \cap B$, $A \setminus B$, $\emptyset$
    \item \textbf{数集}:
    自然数 $\mathbb{N}$, 整数 $\mathbb{Z}$, 有理数 $\mathbb{Q}$, 实数 $\mathbb{R}$, 复数 $\mathbb{C}$
    \item \textbf{逻辑符号}:
    对所有 $\forall$, 存在 $\exists$, 蕴含 $\implies$, 当且仅当 $\iff$, 非 $\neg$
\end{itemize}

% --- 定理环境 ---
\section{定理环境}
使用 \texttt{amsthm} 宏包可以创建结构化的定理、定义等环境。

\begin{definition}[开集]
设 $S \subseteq \mathbb{R}^n$。如果对于任意 $x \in S$,存在 $\epsilon > 0$,使得以 $x$ 为中心、$\epsilon$ 为半径的开球 $B_\epsilon(x) \subset S$,则称 $S$ 为开集。
\end{definition}

\begin{theorem}[中值定理]
设函数 $f$ 在闭区间 $[a, b]$ 上连续,在开区间 $(a, b)$ 内可导,则存在一点 $\xi \in (a, b)$,使得
\[
f'(\xi) = \frac{f(b) - f(a)}{b - a}
\]
\end{theorem}

\begin{proof}
这是微积分中的一个基本定理,证明过程略。
\end{proof}

\begin{example}
区间 $(0, 1)$ 是 $\mathbb{R}$ 上的开集。
\end{example}

\begin{remark}
开集的定义依赖于所处的拓扑空间。
\end{remark}

% --- 多行公式的其他对齐方式 ---
\section{多行公式的其他对齐方式}
除了 \texttt{align},\texttt{amsmath} 还提供了其他对齐环境。

% 使用 gather 显示不需对齐的多行公式
\begin{gather}
e^{i\pi} + 1 = 0 \\
\int_{-\infty}^{\infty} e^{-x^2} dx = \sqrt{\pi} \\
\sum_{n=1}^{\infty} \frac{1}{n^2} = \frac{\pi^2}{6}
\end{gather}

% 使用 multline 显示过长需换行的单个公式
\begin{multline}
(a + b + c + d + e)^2 = a^2 + b^2 + c^2 + d^2 + e^2 \\
+ 2ab + 2ac + 2ad + 2ae + 2bc + 2bd + 2be \\
+ 2cd + 2ce + 2de
\end{multline}

% --- 积分变换与特殊函数 ---
\section{积分变换与特殊函数}
展示傅里叶变换、狄拉克δ函数和伽马函数。

\begin{itemize}
    \item \textbf{傅里叶变换}:
    \[
    \hat{f}(\omega) = \int_{-\infty}^{\infty} f(t) e^{-i\omega t} dt
    \]
    \item \textbf{狄拉克δ函数}:
    \[
    \int_{-\infty}^{\infty} \delta(x) dx = 1, \quad \delta(x) = 0 \text{ for } x \neq 0
    \]
    \item \textbf{伽马函数}:
    \[
    \Gamma(z) = \int_{0}^{\infty} t^{z-1} e^{-t} dt, \quad (z > 0)
    \]
    且 $\Gamma(n+1) = n!$ 对于正整数 $n$。
\end{itemize}

% --- 数学算子与花体 ---
\section{数学算子与花体字母}
定义新的数学算子和使用花体字母。

% 定义新的数学算子
\DeclareMathOperator{\diag}{diag}
\DeclareMathOperator{\rank}{rank}

\begin{itemize}
    \item \textbf{自定义算子}:
    矩阵 $A$ 的对角元素为 $\diag(A)$,其秩为 $\rank(A)$。
    \item \textbf{花体字母}:
    傅里叶变换常记为 $\mathcal{F}\{f(t)\}$,拉普拉斯变换为 $\mathcal{L}\{f(t)\}$。
    \item \textbf{黑板粗体}:
    在定义中已使用 $\mathbb{R}, \mathbb{C}$ 等。
\end{itemize}

% --- 综合应用 ---
\section{综合应用示例}

% 概率密度函数 (正态分布)
\subsection{概率密度函数}
正态分布 $N(\mu, \sigma^2)$ 的概率密度函数为:
\[
f(x) = \frac{1}{\sigma\sqrt{2\pi}} \exp\left(-\frac{(x-\mu)^2}{2\sigma^2}\right)
\]

% 线性回归的正规方程
\subsection{线性回归}
在多元线性回归中,参数 $\boldsymbol{\beta}$ 的最小二乘估计由正规方程给出:
\[
\boldsymbol{\beta} = (\mathbf{X}^T\mathbf{X})^{-1}\mathbf{X}^T\mathbf{y}
\]
其中 $\mathbf{X}$ 是设计矩阵,$\mathbf{y}$ 是响应向量。

\end{document}