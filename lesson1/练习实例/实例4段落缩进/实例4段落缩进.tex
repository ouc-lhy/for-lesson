\documentclass{article}
\usepackage[UTF8]{ctex} % 支持中文

\title{LaTeX 实验报告 - 段落缩进练习}
\author{刘浩洋 24040021022}
\date{2025年8月31日}

% --- 设置全局段落缩进 ---
% 默认情况下,article类的 \parindent 为15pt。这里我们将其设置为2em。
\setlength{\parindent}{2em}

\begin{document}

\maketitle

% --- 默认缩进 ---
\section{默认段落缩进}
在LaTeX中,除了每个章节(section)后的第一个段落外,其余段落的首行会自动缩进。这个缩进量由 \texttt{\textbackslash parindent} 控制。
这是一个新的段落,它的首行会自动缩进2em(因为我们设置了 \texttt{\textbackslash setlength\{\textbackslash parindent\}\{2em\}})。
这是同一段落内的换行,不会缩进。只有新段落的首行才会缩进。

这是另一个新段落,同样会自动缩进。

% --- 取消单个段落缩进 ---
\section{取消单个段落缩进}
如果希望某个特定段落不缩进,可以在该段落开头使用 \texttt{\textbackslash noindent} 命令。
\noindent 这个段落使用了 \texttt{\textbackslash noindent} 命令,因此首行没有缩进。这在需要顶格书写时非常有用,例如在某些格式要求严格的文档中。

这是一个正常的段落,它会继续缩进。

% --- 全局取消缩进 ---
\section{全局取消段落缩进}
如果希望文档中所有段落都不缩进,可以在导言区(preamble)使用 \texttt{\textbackslash setlength\{\textbackslash parindent\}\{0pt\}}。
% 注意:此命令已在导言区设置,此处仅为说明。
% \setlength{\parindent}{0pt} % 如果取消注释并重新编译,后续所有段落将无缩进
% 但为了演示,我们在此处不改变全局设置。

% --- 悬挂缩进 ---
\section{悬挂缩进}
悬挂缩进是指第一行不缩进,而后续行缩进。这可以通过 \texttt{\textbackslash hangindent} 和 \texttt{\textbackslash hangafter} 命令实现。
\hangindent=1cm % 设置悬挂缩进量为1cm
\hangafter=1   % 从第一行之后开始悬挂(即第一行不缩进,后续行缩进)
这是一个实现悬挂缩进的段落。第一行顶格,从第二行开始向右缩进1厘米。这种格式常用于参考文献条目或项目符号列表的描述部分。

这是另一个普通段落,它遵循全局的首行缩进设置。

\end{document}