\documentclass[12pt,a4paper]{report}
\usepackage[UTF8]{ctex} % 支持中文
\usepackage{fancyhdr} % 用于设置页眉页脚
\usepackage{graphicx} % 插入图片
\usepackage{lastpage} % 获取总页数
\usepackage{geometry} % 设置页面边距
\usepackage{xcolor} % 颜色支持

% 设置页面边距
\geometry{left=3cm,right=2.5cm,top=3cm,bottom=2.5cm}

\title{LaTeX页眉页脚}
\author{刘浩洋 24040021022}
\date{2025年9月1日}

% --- 配置 fancyhdr ---
\pagestyle{fancy} % 使用fancy页眉页脚样式
\fancyhf{} % 清空默认的页眉页脚内容

% 设置页眉(适用于所有页面)
\fancyhead[L]{\small\leftmark} % 左侧:当前章节名
\fancyhead[C]{\small LaTeX页眉页脚设置} % 中间:文档标题
\fancyhead[R]{\small\thepage/\pageref{LastPage}} % 右侧:页码/总页数

% 设置页脚(适用于所有页面)
\fancyfoot[L]{\small 刘浩洋 24040021022} % 左侧:作者信息
\fancyfoot[C]{\small \today} % 中间:当前日期
\fancyfoot[R]{\small 第\thepage 页} % 右侧:中文页码格式

% 设置页眉页脚线样式
\renewcommand{\headrulewidth}{0.6pt} % 页眉线宽度
\renewcommand{\headrule}{\hrule width\headwidth height\headrulewidth depth0pt \vskip-\headrulewidth}
\renewcommand{\footrulewidth}{0.4pt} % 页脚线宽度

% 重新定义plain样式(章节首页)
\fancypagestyle{plain}{%
    \fancyhf{} % 清空
    \fancyhead[L]{\small\leftmark} % 保持章节名
    \fancyhead[R]{\small\thepage/\pageref{LastPage}} % 保持页码
    \fancyfoot[L]{\small 刘浩洋 24040021022} % 保持作者信息
    \fancyfoot[C]{\small \today} % 保持日期
    \fancyfoot[R]{\small 第\thepage 页} % 保持页码格式
    \renewcommand{\headrulewidth}{0.6pt} % 保持页眉线
    \renewcommand{\footrulewidth}{0.4pt} % 保持页脚线
}

% 设置章节标题格式
\usepackage{titlesec}
\titleformat{\chapter}[display]
{\normalfont\huge\bfseries}{\chaptertitlename\ \thechapter}{20pt}{\Huge}
\titlespacing*{\chapter}{0pt}{-30pt}{40pt}

\begin{document}

\maketitle

% 标题页使用空页眉页脚
\thispagestyle{empty}

% --- 摘要 ---
\begin{abstract}
\thispagestyle{fancy} % 确保摘要页有页眉页脚
本报告详细演示了如何使用 \texttt{fancyhdr} 宏包自定义LaTeX文档的页眉和页脚。通过设置 \texttt{\textbackslash fancyhead} 和 \texttt{\textbackslash fancyfoot} 命令,可以灵活地在页面的左、中、右位置放置章节标题、页码、文档标题等信息。报告展示了基本配置、线条控制以及为不同页面(如章节首页)设置特殊样式的完整流程。结果表明,\texttt{fancyhdr} 宏包是提升LaTeX文档专业性和美观度的有效工具。
\end{abstract}

% 目录页
\tableofcontents
\thispagestyle{fancy} % 确保目录页有页眉页脚

% --- 正文 ---
\chapter{引言}
LaTeX是一种高质量的排版系统,广泛应用于学术出版和科技文档制作。在撰写长篇文档时,良好的页眉页脚设计不仅能提升文档的专业性,还能方便读者导航和定位内容。

本报告将详细介绍如何使用fancyhdr宏包自定义LaTeX文档的页眉和页脚设置,包括基本配置方法和高级定制技巧。

\section{研究背景}
随着学术交流的日益频繁,规范的文档格式显得尤为重要。页眉页脚作为文档的重要组成部分,不仅承载着页码信息,还可以展示章节标题、作者信息、文档属性等内容。

传统的LaTeX默认页眉页脚样式较为简单,无法满足多样化需求。fancyhdr宏包应运而生,为用户提供了灵活且强大的页眉页脚定制功能。

\section{研究目标}
本研究旨在探索fancyhdr宏包的核心功能和使用方法,具体目标包括:

1. 掌握fancyhdr宏包的基本命令和配置方式
2. 学习如何在文档不同部分设置不同的页眉页脚样式
3. 了解页眉页脚线条的自定义方法
4. 实现一个完整的页眉页脚定制案例

\chapter{fancyhdr 宏包详解}
fancyhdr宏包是LaTeX中用于自定义页眉页脚的最常用工具之一。它提供了丰富的命令和选项,允许用户在文档的奇数页、偶数页以及不同章节中设置不同的页眉页脚内容。

\section{基本命令}
fancyhdr宏包的核心命令包括:\texttt{\textbackslash fancyhead}、\texttt{\textbackslash fancyfoot}和\texttt{\textbackslash fancyhf}。这些命令接受三个位置参数:[L](左)、[C](中)和[R](右),用于指定相应位置显示的内容。

用户可以在这些位置插入静态文本、动态标记(如章节标题)、页码或自定义命令输出。

\section{高级配置}
除了基本的内容设置外,fancyhdr还支持多种高级配置选项:

1. 页眉页脚线条样式定制:可以调整线条的粗细、颜色甚至完全隐藏
2. 不同页面样式定义:可以为普通页、章节首页等设置不同的页眉页脚
3. 奇偶页差异化设计:可以分别为奇数页和偶数页设置不同的布局
4. 字段宽度调整:可以控制左、中、右三个区域的宽度比例

通过这些高级功能,用户可以创建出既美观又实用的页眉页脚设计,大大提升文档的专业程度和可读性。

\chapter{实际应用案例}
在实际文档制作中,合理的页眉页脚设置能够显著提升文档质量。以下是一些常见场景的应用示例。

\section{学术论文格式}
学术论文通常要求在页眉中包含论文标题或简短标题,页脚中包含页码和日期。使用fancyhdr可以轻松实现这一要求。

\section{技术报告格式}
技术报告可能需要在页眉中显示章节名称,在页脚中显示公司logo或保密级别信息。fancyhdr提供了足够的灵活性来满足这些需求。

\section{书籍排版}
书籍排版通常需要区分奇偶页,奇数页显示章节标题,偶数页显示书名。fancyhdr的奇偶页差异化功能非常适合这种场景。

\end{document}