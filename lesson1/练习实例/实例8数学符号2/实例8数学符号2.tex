\documentclass{article}
\usepackage[UTF8]{ctex} % 支持中文

\title{LaTeX数学符号练习}
\author{刘浩洋 24040021022}
\date{2025年8月31日}

\begin{document}

\maketitle

% --- 求和符号 ---
\section{求和符号}
使用 \texttt{\textbackslash sum} 命令表示求和。下标表示求和的起始值,上标表示结束值。

\begin{itemize}
    \item \textbf{行内求和}:$ \sum_{i=1}^{n} i = \frac{n(n+1)}{2} $,这是一个等差数列求和公式。
    \item \textbf{独立求和}:
    \[
    \sum_{k=0}^{\infty} ar^k = \frac{a}{1-r} \quad (|r| < 1)
    \]
    这是无穷等比数列求和公式。
    \item \textbf{上下限位置}:在独立公式中,上下限默认在符号正上方和正下方。在行内公式中,为了节省垂直空间,上下限显示在右上角和右下角。可以使用 \texttt{\textbackslash limits} 强制在行内也显示在正上下方:$ \sum\limits_{i=1}^{n} x_i $。
\end{itemize}

% --- 积分符号 ---
\section{积分符号}
使用 \texttt{\textbackslash int} 命令表示积分。同样使用下标和上标指定积分上下限。

\begin{itemize}
    \item \textbf{定积分}:
    \[
    \int_{0}^{1} x^2 dx = \left[ \frac{x^3}{3} \right]_{0}^{1} = \frac{1}{3}
    \]
    \item \textbf{不定积分}:$ \int x^2 dx = \frac{x^3}{3} + C $
    \item \textbf{多重积分}:
    \[
    \iint\limits_{D} f(x,y) \, dxdy, \quad \iiint\limits_{V} \rho \, dV
    \]
    \item \textbf{环路积分}:$ \oint_C \mathbf{F} \cdot d\mathbf{r} $
\end{itemize}

% --- 希腊字母 ---
\section{希腊字母}
LaTeX提供了完整的希腊字母命令。小写字母命令如 \texttt{\textbackslash alpha}, \texttt{\textbackslash beta};大写字母命令如 \texttt{\textbackslash Gamma}, \texttt{\textbackslash Delta}。

\begin{itemize}
    \item \textbf{常用小写}:
    $\alpha$, $\beta$, $\gamma$, $\delta$, $\epsilon$, $\zeta$, $\eta$, $\theta$, $\iota$, $\kappa$, $\lambda$, $\mu$, $\nu$, $\xi$, $\pi$, $\rho$, $\sigma$, $\tau$, $\upsilon$, $\phi$, $\chi$, $\psi$, $\omega$
    \item \textbf{常用大写}:
    $\Gamma$, $\Delta$, $\Theta$, $\Lambda$, $\Xi$, $\Pi$, $\Sigma$, $\Upsilon$, $\Phi$, $\Psi$, $\Omega$
    \item \textbf{组合应用}:
    $ e^{i\pi} + 1 = 0 $ (欧拉公式),$ \int_{-\infty}^{\infty} e^{-x^2/2} dx = \sqrt{2\pi} $
\end{itemize}

% --- 综合应用 ---
\section{综合应用示例}
将求和、积分和希腊字母组合使用,表达更复杂的数学概念。
\[
\mathcal{F}\{f(t)\} = F(\omega) = \int_{-\infty}^{\infty} f(t) e^{-i\omega t} dt
\]
这是连续傅里叶变换的定义式,其中使用了积分、希腊字母 $\omega$ (omega) 和虚数单位 $i$。

\end{document}