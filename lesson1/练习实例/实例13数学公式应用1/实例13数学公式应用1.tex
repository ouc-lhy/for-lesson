\documentclass{article}
\usepackage[UTF8]{ctex} % 支持中文
\usepackage{amsmath} % 数学公式核心宏包
\usepackage{amssymb} % 额外数学符号

\title{LaTeX数学公式综合练习}
\author{刘浩洋 24040021022}
\date{2025年9月1日}

\begin{document}

\maketitle

% --- 矩阵 ---
\section{矩阵}
使用 \texttt{amsmath} 宏包提供的环境可以轻松创建矩阵。

\begin{itemize}
    \item \textbf{无括号矩阵}:
    \[
    \begin{matrix}
    a & b \\
    c & d 
    \end{matrix}
    \]
    \item \textbf{方括号矩阵} (常用):
    \[
    A = \begin{bmatrix}
    1 & 2 & 3 \\
    4 & 5 & 6 \\
    7 & 8 & 9 
    \end{bmatrix}
    \]
    \item \textbf{圆括号矩阵}:
    \[
    \begin{pmatrix}
    x_1 \\
    x_2 \\
    x_3 
    \end{pmatrix}
    \]
\end{itemize}

% --- 多行公式 ---
\section{多行公式}
使用 \texttt{align} 环境可以对齐多行公式,通常在等号处对齐。

\begin{align}
(a + b)^2 &= a^2 + 2ab + b^2 \\
(a - b)^2 &= a^2 - 2ab + b^2 \\
(a + b)(a - b) &= a^2 - b^2
\end{align}

% 使用 \texttt{align*} 环境可以不编号
\begin{align*}
\sin(2x) &= 2\sin x \cos x \\
\cos(2x) &= \cos^2 x - \sin^2 x \\
&= 2\cos^2 x - 1 \\
&= 1 - 2\sin^2 x
\end{align*}

% --- 条件定义 ---
\section{条件定义}
使用 \texttt{cases} 环境可以定义分段函数。

\[
|x| = \begin{cases} 
x & \text{if } x \geq 0 \\
-x & \text{if } x < 0 
\end{cases}
\]

\[
f(n) = \begin{cases} 
n/2 & \text{if } n \text{ is even} \\
3n+1 & \text{if } n \text{ is odd} 
\end{cases}
\]

% --- 微积分与向量 ---
\section{微积分与向量}
展示极限、偏导数、梯度和向量运算。

\begin{itemize}
    \item \textbf{极限}:
    \[
    \lim_{x \to 0} \frac{\sin x}{x} = 1
    \]
    \item \textbf{偏导数}:
    设 $f(x,y) = x^2y + \sin(xy)$,则
    \[
    \frac{\partial f}{\partial x} = 2xy + y\cos(xy), \quad
    \frac{\partial f}{\partial y} = x^2 + x\cos(xy)
    \]
    \item \textbf{梯度}:
    \[
    \nabla f = \left( \frac{\partial f}{\partial x}, \frac{\partial f}{\partial y} \right)
    \]
    \item \textbf{向量点积与叉积}:
    \[
    \mathbf{a} \cdot \mathbf{b} = |\mathbf{a}| |\mathbf{b}| \cos \theta, \quad
    \mathbf{a} \times \mathbf{b} = |\mathbf{a}| |\mathbf{b}| \sin \theta \, \mathbf{n}
    \]
\end{itemize}

% --- 自动调整括号大小 ---
\section{自动调整括号大小}
使用 \texttt{\textbackslash left} 和 \texttt{\textbackslash right} 可以让括号(或其它分隔符)根据内容自动调整大小。

\[
\left( \frac{a+b}{c} \right)^2, \quad
\left[ \sum_{i=1}^{n} \left( x_i - \bar{x} \right)^2 \right]^{\frac{1}{2}}
\]

% --- 综合应用 ---
\section{综合应用示例}
将多种数学结构组合,表达复杂的数学概念。

% 麦克斯韦方程组(积分形式)
\subsection{麦克斯韦方程组}
\begin{align}
\oint_{\partial \Omega} \mathbf{E} \cdot d\mathbf{l} &= -\frac{d}{dt} \iint_{\Omega} \mathbf{B} \cdot d\mathbf{S} \\
\oint_{\partial \Omega} \mathbf{B} \cdot d\mathbf{l} &= \mu_0 \iint_{\Omega} \mathbf{J} \cdot d\mathbf{S} + \mu_0\epsilon_0 \frac{d}{dt} \iint_{\Omega} \mathbf{E} \cdot d\mathbf{S}
\end{align}

% 线性方程组
\subsection{线性方程组}
\[
\begin{bmatrix}
2 & 1 & -1 \\
-3 & -1 & 2 \\
-2 & 1 & 2 
\end{bmatrix}
\begin{bmatrix}
x \\
y \\
z 
\end{bmatrix}
=
\begin{bmatrix}
8 \\
-11 \\
-3 
\end{bmatrix}
\]

\end{document}