\documentclass{article}
\usepackage[UTF8]{ctex} % 支持中文

\title{LaTeX 实验报告 - 数学符号练习}
\author{刘浩洋 24040021022}
\date{2025年8月31日}

\begin{document}

\maketitle

% --- 上标和下标 ---
\section{上标和下标}
在数学模式中,使用 \texttt{\^{}} 表示上标,\texttt{\_} 表示下标。如果上标或下标包含多个字符,需要用花括号 \texttt{\{...\}} 括起来。

\begin{itemize}
    \item \textbf{简单上标}:$x^2$, $e^x$, $a^n$
    \item \textbf{复杂上标}:$e^{x^2 + y^2}$, $2^{n+1}$
    \item \textbf{简单下标}:$x_1$, $a_i$, $y_{max}$
    \item \textbf{复杂下标}:$a_{ij}$, $x_{n+1}$
    \item \textbf{上下标组合}:$x_1^2$, $a_{ij}^{(k)}$
\end{itemize}

% --- 根号 ---
\section{根号}
使用 \texttt{\textbackslash sqrt} 命令生成根号。对于n次方根,使用 \texttt{\textbackslash sqrt[n]\{...\}}。

\begin{itemize}
    \item \textbf{平方根}:$\sqrt{2}$, $\sqrt{x^2 + y^2}$
    \item \textbf{立方根}:$\sqrt[3]{8} = 2$
    \item \textbf{n次方根}:$\sqrt[n]{a}$, $\sqrt[5]{32} = 2$
\end{itemize}

% --- 分数 ---
\section{分数}
使用 \texttt{\textbackslash frac\{分子\}\{分母\}} 命令生成分数。在行内公式和独立公式中显示效果略有不同。

\begin{itemize}
    \item \textbf{行内分数}:一个简单的分数是 $\frac{1}{2}$,另一个是 $\frac{a+b}{c}$。
    \item \textbf{独立分数}:
    \[
    \frac{x^2 - 1}{x + 1} = x - 1 \quad (x \neq -1)
    \]
    \item \textbf{复杂分数}:
    \[
    \frac{a + \frac{b}{c}}{d} = \frac{a}{d} + \frac{b}{cd}
    \]
\end{itemize}

\end{document}