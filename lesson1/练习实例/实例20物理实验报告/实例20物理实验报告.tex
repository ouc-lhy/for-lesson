\documentclass[12pt,a4paper]{article}
\usepackage[UTF8]{ctex} % 支持中文
\usepackage{graphicx} % 插入图片
\usepackage{fancyhdr} % 自定义页眉页脚
\usepackage[colorlinks, linkcolor=blue, urlcolor=red]{hyperref} % 超链接
\usepackage{lipsum} % 生成示例文本

\title{物理实验报告\\\large 牛顿第二定律的验证}
\author{刘浩洋 24040021022}
\date{2025年9月1日}

% --- 配置页眉页脚 ---
\pagestyle{fancy}
\fancyhf{}
\fancyhead[L]{\leftmark} % 左侧显示章节名
\fancyhead[R]{\thepage} % 右侧显示页码
\renewcommand{\headrulewidth}{0.4pt} % 页眉线
\fancypagestyle{plain}{% 重新定义章节首页样式
    \fancyhf{}
    \fancyfoot[C]{\thepage}
    \renewcommand{\headrulewidth}{0pt}
}

\begin{document}

\maketitle

% --- 摘要 ---
\begin{abstract}
本实验旨在通过测量不同外力作用下物体的加速度,验证牛顿第二定律 $F = ma$ 的正确性。实验采用气垫导轨减小摩擦力,利用光电门和数字计时器测量滑块通过固定距离的时间,从而计算加速度。通过改变悬挂砝码的质量来改变外力 $F$,并保持滑块总质量 $m$ 不变。实验数据表明,外力 $F$ 与加速度 $a$ 呈线性关系,线性拟合的斜率即为滑块的总质量 $m$。实验结果与理论预测相符,验证了牛顿第二定律。
\end{abstract}

\tableofcontents
\newpage % 目录后换页

% --- 正文 ---
\section{引言}
牛顿第二定律是经典力学的基础,其数学表达式为:
\begin{equation}
    F = ma
\end{equation}
其中,$F$ 为物体所受合外力,$m$ 为物体质量,$a$ 为物体加速度。本实验通过控制变量法,验证该定律。

\section{实验原理}
\subsection{理论依据}
根据牛顿第二定律,当物体质量 $m$ 保持不变时,物体的加速度 $a$ 与所受合外力 $F$ 成正比,即 $a \propto F$。

\subsection{实验方法}
实验装置如图\ref{fig:setup}所示。通过改变悬挂砝码的质量 $m_g$ 来改变外力 $F \approx m_g g$(忽略滑轮摩擦和细线质量)。利用公式 $a = \frac{2s}{t^2}$ 计算加速度,其中 $s$ 为滑块运动的固定距离,$t$ 为通过该距离的时间。

\section{实验装置与步骤}
\subsection{实验装置}
主要装置包括:
\begin{itemize}
    \item 气垫导轨及滑块
    \item 光电门和数字计时器
    \item 砝码组
    \item 细线和滑轮
\end{itemize}

\begin{figure}[htbp]
    \centering
    % \includegraphics[width=0.6\textwidth]{setup.jpg} % 实际使用时替换为真实图片路径
    \framebox{\begin{minipage}{8cm} % 此处用一个方框代替真实图片,用于演示
        \centering
        实验装置示意图 (此处应插入图片)
    \end{minipage}}
    \caption{实验装置示意图}
    \label{fig:setup}
\end{figure}

\subsection{实验步骤}
\begin{enumerate}
    \item 调节气垫导轨水平。
    \item 测量滑块和附加砝码的总质量 $m$。
    \item 固定光电门位置,测量距离 $s$。
    \item 将质量为 $m_g$ 的砝码挂在细线一端,另一端连接滑块。
    \item 打开气源,释放滑块,记录通过距离 $s$ 的时间 $t$。
    \item 改变 $m_g$,重复步骤5,获取多组数据。
\end{enumerate}

\section{数据记录与处理}
\subsection{数据记录}
实验数据如下表所示:

% 简单表格示例
\begin{center}
\begin{tabular}{|c|c|c|c|}
\hline
$m_g$ (g) & $F \approx m_g g$ (N) & $t$ (s) & $a = 2s/t^2$ (m/s²) \\
\hline
10 & 0.098 & 1.42 & 0.996 \\
20 & 0.196 & 1.00 & 2.000 \\
30 & 0.294 & 0.82 & 2.951 \\
40 & 0.392 & 0.71 & 3.972 \\
\hline
\end{tabular}
\end{center}

\subsection{数据处理}
根据表中数据,绘制 $F-a$ 关系图。使用线性拟合,得到拟合方程 $F = ka$,其中斜率 $k$ 应等于滑块的总质量 $m$。将拟合得到的 $k$ 与实验测量的 $m$ 进行比较,计算相对误差。

\section{结果与讨论}
实验测得滑块总质量 $m_{\text{测}} = 0.400 \, \text{kg}$。$F-a$ 图的线性拟合斜率 $k = 0.397 \, \text{kg}$。相对误差为:
\begin{equation}
    E = \frac{|m_{\text{测}} - k|}{m_{\text{测}}} \times 100\% = \frac{|0.400 - 0.397|}{0.400} \times 100\% = 0.75\%
\end{equation}
误差较小,说明实验结果与牛顿第二定律相符。

\section{结论}
通过本实验,我们成功验证了牛顿第二定律 $F = ma$。在质量一定的情况下,物体的加速度与所受合外力成正比。实验数据的线性关系良好,相对误差较小,达到了实验预期目标。

\section{参考文献}
\begin{enumerate}
    \item \url{https://zh.wikipedia.org/wiki/牛顿第二定律}
    \item 张三, 李四. 大学物理实验教程[M]. 北京: 高等教育出版社, 2020.
\end{enumerate}

\end{document}