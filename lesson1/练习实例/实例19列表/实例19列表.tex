\documentclass{article}
\usepackage[UTF8]{ctex} % 支持中文
\usepackage{lipsum} % 生成示例文本

\title{LaTeX 实验报告 - 列表环境使用}
\author{刘浩洋 24040021022}
\date{2025年9月1日}

\begin{document}

\maketitle

\section{引言}
在撰写文档时,将信息以列表形式呈现是提高可读性和条理性的重要手段。LaTeX提供了简单而强大的列表环境,主要包括无序的项目列表(`itemize`)和有序的编号列表(`enumerate`)。正确使用列表环境可以使文档结构清晰,重点突出。

\section{项目列表 (itemize)}
项目列表用于罗列没有特定顺序的条目,每个条目前面通常有一个项目符号(如圆点)。

\subsection{基本语法}
使用 `\texttt{\textbackslash begin\{itemize\}}` 和 `\texttt{\textbackslash end\{itemize\}}` 开始和结束一个项目列表。在列表内部,使用 `\texttt{\textbackslash item}` 命令来标记每一项。

\subsection{示例}
以下是一个关于LaTeX优点的项目列表:

\begin{itemize}
    \item 输出质量高,排版精美。
    \item 数学公式输入强大且规范。
    \item 分离内容与格式,便于修改整体风格。
    \item 开源免费,社区支持广泛。
    \item 特别适合撰写科技论文、报告和书籍。
\end{itemize}

\section{编号列表 (enumerate)}
编号列表用于罗列有特定顺序或需要强调先后次序的条目,每个条目前面有数字或字母编号。

\subsection{基本语法}
使用 `\texttt{\textbackslash begin\{enumerate\}}` 和 `\texttt{\textbackslash end\{enumerate\}}` 开始和结束一个编号列表。同样使用 `\texttt{\textbackslash item}` 命令来标记每一项。

\subsection{示例}
以下是使用LaTeX编写文档的基本步骤:

\begin{enumerate}
    \item 安装LaTeX发行版(如TeX Live或MiKTeX)。
    \item 选择一个编辑器(如TeXworks, VS Code, 或在线平台Overleaf)。
    \item 创建一个 `.tex` 文件。
    \item 在文件中编写LaTeX代码,包括文档类、宏包、标题、作者和正文。
    \item 使用编译器(如XeLaTeX或PDFLaTeX)将 `.tex` 文件编译成PDF。
    \item 检查PDF输出,根据需要修改源代码并重新编译。
\end{enumerate}

\section{列表嵌套}
LaTeX允许将 `itemize` 和 `enumerate` 环境相互嵌套,以创建层次化的信息结构。

\subsection{示例}
以下是一个嵌套列表,展示了一个学习计划:

\begin{enumerate}
    \item 学习LaTeX基础
        \begin{itemize}
            \item 掌握文档结构
            \item 学习文本格式化
            \item 练习插入图片和表格
        \end{itemize}
    \item 深入数学公式
        \begin{enumerate}
            \item 基本数学模式
            \item 矩阵与多行公式
            \item 定理环境
        \end{enumerate}
    \item 探索高级功能
        \begin{itemize}
            \item 绘制流程图 (TikZ)
            \item 制作演示文稿 (Beamer)
            \item 管理参考文献 (BibTeX)
        \end{itemize}
\end{enumerate}

\section{结论}
通过本次练习,我掌握了LaTeX中 `itemize` 和 `enumerate` 两种列表环境的使用方法。项目列表适用于罗列并列的观点或特性,而编号列表则适用于描述有先后顺序的步骤或流程。更重要的是,列表的嵌套功能使得构建复杂的、层次分明的信息结构成为可能。合理使用列表是提升文档专业性和可读性的基础技能,对于撰写清晰的报告和论文至关重要。

\end{document}