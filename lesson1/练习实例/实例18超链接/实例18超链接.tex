\documentclass{article}
\usepackage[UTF8]{ctex} % 支持中文
\usepackage[colorlinks, linkcolor=blue, urlcolor=red]{hyperref} % 使用 hyperref 宏包添加超链接支持
% colorlinks: 使链接带有颜色而非边框
% linkcolor: 设置内部链接的颜色(如目录、引用)
% urlcolor: 设置网址链接的颜色

\title{LaTeX 实验报告 - 超链接使用}
\author{刘浩洋 24040021022}
\date{2025年9月1日}

\begin{document}

\maketitle

\section{引言}
在现代科技文献中,超链接是非常重要的组成部分。它们可以方便地将读者引导至相关的在线资源或文档内的特定部分。LaTeX通过 `hyperref` 宏包提供了强大的超链接功能。本文将介绍如何在LaTeX文档中使用超链接。

\section{基本超链接}
使用 `hyperref` 宏包提供的命令,可以轻松地插入网页链接和电子邮件地址。

\subsection{网页链接}
要创建一个网页链接,可以使用 `\href{URL}{text}` 命令。例如,点击 \href{https://www.latex-project.org/}{这里} 访问 LaTeX 官方网站。

或者,如果只需要显示 URL,可以使用 `\url{URL}` 命令,比如访问 \url{https://www.overleaf.com/}。

\subsection{邮件链接}
对于电子邮件链接,同样可以使用 `\href` 命令。例如,发送邮件给 \href{mailto:example@example.com}{示例邮箱}。

\section{内部超链接}
除了外部链接,`hyperref` 还允许在同一文档内创建指向其他位置的链接。这通常用于长篇文档中,帮助读者快速导航到相关内容。

\subsection{标签与引用}
首先,在需要链接到的位置放置一个标签,使用 `\label{marker}`。然后,在文档的任何地方使用 `\ref{marker}` 或 `\pageref{marker}` 来引用该位置。`hyperref` 会自动将其转换为可点击的链接。

例如,见第\ref{sec:example}节“示例”(位于第\pageref{sec:example}页)。

\subsection{显式超链接}
更灵活的方法是使用 `\hypertarget` 和 `\hyperlink` 命令来创建和链接到特定位置。

\hypertarget{mytarget}{这是一个目标位置}。你可以从文档的任何地方通过以下链接跳转到这里:\hyperlink{mytarget}{跳转到目标位置}。

\section{示例}
\label{sec:example}
这里是文档的一个示例部分,它被标记了标签 `sec:example`。如前面所述,可以通过 `\ref{sec:example}` 或者直接使用超链接 `\hyperlink{mytarget}{跳转到目标位置}` 来访问。

\section{结论}
通过使用 `hyperref` 宏包,LaTeX 文档能够包含丰富的超链接,从而增强文档的互动性和实用性。无论是外部链接还是内部导航,都可以极大地改善读者的阅读体验。掌握这些技巧,可以使你的文档更加专业和用户友好。

\end{document}