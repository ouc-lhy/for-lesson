\documentclass{article} % 使用article文档类
\usepackage[UTF8]{ctex} % 引入ctex宏包以支持中文

\title{LaTeX文档结构练习}
\author{刘浩洋 24020021022}
\date{2025年8月31日}

\begin{document}

\maketitle % 生成标题

% --- 正文开始 ---
\section{引言}
这是文档的第一级章节(Section)。它通常用于划分文档的主要部分。例如,引言、方法、结果、讨论等。

\subsection{研究背景}
这是第二级章节(Subsection),用于在一级章节下进一步细分内容。它比一级章节低一个层级。

\subsubsection{具体问题}
这是第三级章节(Subsubsection),层级更深,用于组织更细致的内容。在实际写作中,应避免层级过深,以免结构混乱。

\paragraph{一个要点}
这是第四级标题(Paragraph)。它在文档中通常以粗体行开始,不单独成行编号,用于强调段落中的一个重点。

\subparagraph{补充说明}
这是第五级标题(Subparagraph),与第四级类似,但层级更低。它同样以粗体行开始,用于更细的补充说明。

\section{实验方法}
这是另一个一级章节。通过使用不同的章节命令,我们可以清晰地构建文档的逻辑框架。

\subsection{数据收集}
描述数据是如何收集的。

\subsection{数据分析}
描述数据分析所采用的方法。

\section{结论}
总结本次练习的主要收获。

\end{document}